\documentclass[unicode]{beamer}

\usepackage{array} % for m column type
\usepackage{amsmath} % align
\usepackage{bbm}
\usepackage{booktabs}
\usepackage{csvsimple} % for \csvreader
\usepackage[utf8]{inputenc}
\usepackage[orientation=portrait,size=a0,scale=1.4]{beamerposter}
\usepackage{graphicx}

\usepackage[style=numeric, sorting=none, url=false, doi=false]{biblatex}
\addbibresource{src/ref.bib}
\AtBeginBibliography{
  \scriptsize
	\setlength{\labelsep}{0pt}
	\renewcommand{\arraystretch}{0.2}
}

\usepackage{tikz}
\usetikzlibrary{
  arrows.meta, % for changing arrow head size
  calc,
  positioning,
  shapes.callouts,
  tikzmark,
}

\input{src/tex/macro}

\newcommand{\dtrain}{D_{\mathrm{train}}}
\newcommand{\dtest}{D_{\mathrm{test}}}
\newcommand{\dmis}{D_{\mathrm{mis}}}
\newcommand{\dchange}{D_{\mathrm{change}}}

\setbeamercolor{hokudai}{bg=hokudai!20}

\usetheme[footertext={
  27th International Conference on Pattern Recognition, December 01--05, 2024,
  Kolkata, India
  }]{SimplePoster}

\title{
  R-LIME:Rectangular Constraints and Optimization for \\
  Local Interpretable Model-agnostic Explanation Methods
}
\author{Genji Ohara, Keigo Kimura, Mineichi Kudo}
\institute{
  Division of Computer Science and Information Technology \\
  Graduate School of Information Sci\@. and Tech., Hokkaido University
}
\date{\today}

\begin{document}

\begin{frame}
	\vspace{-0.2em}
	\begin{columns}[t]
		\def\lcol{0.375}
		\def\rcol{0.62}
		\hspace{-1.0em}
		\begin{column}{\lcol\linewidth}
			\begin{block}{1. Background}
				\hspace{-0.5em}
				\begin{beamercolorbox}[wd=.52\textwidth,colsep=0.3cm,rounded=true,shadow=true]{hokudai}
					Interpretable Machine Learning
				\end{beamercolorbox}
				\bigskip
				\begin{columns}[]
					\begin{column}{0.4\textwidth}
						{
							\begin{itemize}
								\item simple models
								      \begin{itemize}
									      \item linear models
									      \item decision trees
								      \end{itemize}
							\end{itemize}
						}
						\hspace{0.5em}
						\textrightarrow{} process is \underline{clear}
					\end{column}
					\begin{column}{0.4\textwidth}
						\begin{itemize}
							\item complex models
							      \begin{itemize}
								      \item deep neural networks
								      \item ensemble models
							      \end{itemize}
						\end{itemize}
						\hspace{0.5em}
						\textrightarrow{} process is \underline{unclear}
					\end{column}
				\end{columns}
				\vspace{0.6em}
				\begin{center}
					\underline{Locally approximate complex models by simple models}
				\end{center}
			\end{block}
			\vspace{-0.6em}
			\begin{block}{2. Related Work}
				\hspace{-1.0em}
				\textbf{LIME \small(Local Interpretable Model-agnostic Explanations)}\cite{ribeiro2016why}
				\vspace{-0.6em}
				\begin{columns}
					\begin{column}{.55\textwidth}
						{
							\renewcommand{\leftmargini}{2.5em}
							\begin{enumerate}
								\item Sample perturbed instances around the given focal point
								\item Learn a linear model on the instances
							\end{enumerate}
						}
					\end{column}
					\begin{column}{.45\textwidth}
						\begin{figure}
							\centering
							\includegraphics[width=.8\textwidth]{src/img/visual-lime}
						\end{figure}
					\end{column}
				\end{columns}

				\vspace{1.0em}
				\hspace{-1.0em}
				\textbf{Anchor}\cite{ribeiro2018anchors}
				\vspace{-0.5em}
				\begin{columns}
					\begin{column}{.55\textwidth}
						{
							\renewcommand{\leftmargini}{2.5em}
							\begin{enumerate}
								\item Maximize the rectangular region as long as
								      the model’s outputs are consistent with high probability
							\end{enumerate}
						}
					\end{column}
					\begin{column}{.45\textwidth}
						\begin{figure}
							\centering
							\includegraphics[width=.8\textwidth]{src/img/visual-anchor}
						\end{figure}
					\end{column}
				\end{columns}
				\vspace{1.0em}
				\hspace{-1.0em}
				\textbf{LIME vs. Anchor}
				\vspace{0.5em}
				\begin{columns}[]
					\begin{column}{0.4\textwidth}
						\begin{figure}
							\includegraphics[width=\textwidth]{src/img/example-instance}
							\caption{The focal point}
						\end{figure}
						\vspace{0.5em}
						\begin{figure}
							\includegraphics[width=\textwidth]{src/img/example-anchor}
							\caption{Anchor's explanation}
						\end{figure}
					\end{column}
					\begin{column}{0.35\textwidth}
						\begin{figure}
							\includegraphics[width=\textwidth]{src/img/example-lime}
							\caption{LIME's explanation}
						\end{figure}
					\end{column}
				\end{columns}
				\vspace{0.3em}
				{
					\begin{center}~%
						\begin{beamercolorbox}[wd=0.7\textwidth,colsep=0.3cm,rounded=true,shadow=true]{hokudai}
							\vspace{-0.2em}
							\begin{itemize}
								\item LIME\@: unclear and not optimal scope
								\item Anchor: users get less insight
							\end{itemize}
							\vspace{0.1em}
						\end{beamercolorbox}~%
					\end{center}
				}
				\vspace{-0.25em}
			\end{block}
		\end{column}
		\begin{column}{\rcol\textwidth}
			\begin{block}{3. Proposed Method}
				\begin{columns}[]
					\begin{column}{0.6\textwidth}
						\begin{beamercolorbox}[wd=.77\textwidth,colsep=0.3cm,rounded=true,shadow=true]{hokudai}
							\textbf{R-LIME (Ruled LIME)} = LIME + Anchor
						\end{beamercolorbox}
						\vspace{0.3em}
						\begin{itemize}
							\item Approximate in rectangular region
							\item Maximize the region as long as approximation accuracy is
							      higher than the given threshold
							\item Express the region as a conjunction of feature predicates \\[0.5em]
							      \hspace{0.2em}\small{
								      ex. $\textrm{Gender} = \textrm{'Male' AND } 20\le\textrm{Age} < 30$
							      }
						\end{itemize}
					\end{column}
					\begin{column}{0.3\textwidth}
						\begin{figure}
							\includegraphics[width=\textwidth]{src/img/visual-rlime3}
						\end{figure}
					\end{column}
				\end{columns}

				\vspace{1.0em}
				\begin{columns}[t]
					\def\lcol{0.50}
					\def\rcol{0.48}
					\begin{column}{\lcol\textwidth}
						\underline{\textbf{Settings}}
						\begin{align*}
							 & \text{input space (discretized)}   &  & \ispace             \\
							 & \text{a black-box classifier}      &  & f:\ispace\to\{0,1\} \\
							 & \text{a focal point}               &  & x\in\ispace         \\
							 & \text{distribution on input space} &  & \mathcal{D}         \\
							 & \text{all possible approx.\ model} &  & G
						\end{align*}

						\vspace{0.8em}
						\underline{\textbf{Definitions}}

						\vspace{1.0em}
						\begin{columns}
							\begin{column}{0.95\textwidth}
								rule: a conjunction of predicates
								\begin{align*}
									A(z)   & =a_{i_1}(z)\wedge a_{i_2}(z)\wedge\dots\wedge a_{i_k}(z) \\
									a_i(z) & =\mathbbm{1}_{z_i=x_i}
								\end{align*}
								accuracy: expected accuracy of approx.\ $g$ in $A$
								\begin{equation*}
									\Prec(A)=\underset{g\in G}{\max}
									\ \mathbb{E}_{z\sim\mathcal{D}(z|A)}[\mathbbm{1}_{f(z)=g(z)}]
								\end{equation*}
								coverage: probability that sample $z$ is inside $A$
								\begin{equation*}
									\operatorname{cov}(A)=
									\mathbb{E}_{z\sim\mathcal{D}(z)}[A(z)]
								\end{equation*}
								our problem:

								\vspace{0.2em}
								\begin{center}~%
									\begin{beamercolorbox}[wd=.85\textwidth,colsep=0.1cm,rounded=true,shadow=true]{hokudai}
										\begin{center}
											$
												\tilde{A}=\underset{A\ s.t.
													\ P(\Prec(A)\ge\tau)\ge1-\delta,A(x)=1}  % chktex 36
												{\arg\max}\operatorname{cov}(A)  % chktex 36
											$
										\end{center}~%
									\end{beamercolorbox}

									\vspace{0.5em}
									\underline{Maximize coverage under constraint of accuracy}
								\end{center}
								\vspace{0.8em}
								\begin{figure}[t]
									\centering
									\includegraphics[width=0.3\textwidth]{src/img/visual-rlime1}
									\includegraphics[width=0.3\textwidth]{src/img/visual-rlime2}
									\includegraphics[width=0.3\textwidth]{src/img/visual-rlime3}
								\end{figure}
							\end{column}
						\end{columns}
					\end{column}
					\begin{column}{\rcol\textwidth}
						\hspace{-1.0em}
						\underline{\textbf{Algorithm}} (beam search)

						\vspace{1.0em}
						$\mathcal{A}_{t-1}=\{A_1,\ldots,A_B\}$
						\begin{tikzpicture}
							\node (a) {};
							\node [below = 6cm of a] (b) {};
							\draw [-{Latex[length=5mm, width=5mm]}] (a) to node[right] {
									\begin{beamercolorbox}[wd=.8\textwidth,colsep=0.2cm,rounded=true,shadow=true]{hokudai}
										Generate a set of candidate rules
										\small
										\begin{itemize}
											\item add a new predicate to each rule\\
											      $\mathcal{A}_{t-1}=\{a_1\}$\\
											      $\rightarrow\mathcal{A}_t=\{a_1\wedge a_2,
												      a_1\wedge a_3,a_1\wedge a_4,\dots\}$
										\end{itemize}
									\end{beamercolorbox}
								} (b);
						\end{tikzpicture}
						$\bar{\mathcal{A}}_{t}=\{A_1\wedge a_1,A_1\wedge a_2,\ldots,A_B\wedge a_m\}$
						\begin{tikzpicture}
							\node (a) {};
							\node [below = 6cm of a] (b) {};
							\draw [-{Latex[length=5mm, width=5mm]}] (a) to node[right] {
									\begin{beamercolorbox}[wd=.8\textwidth,colsep=0.2cm,rounded=true,shadow=true]{hokudai}
										Search rules with highest accuracy
										\small
										\begin{itemize}
											\item solve as best arm identification in multi-armed
											      bandit problem using KL-LUCB algorithm\cite{kaufmann2013information}
										\end{itemize}
									\end{beamercolorbox}
								} (b);
						\end{tikzpicture}
						$\mathcal{A}_t=\{A'_1,A'_2,\ldots,A'_B\}$
						\begin{tikzpicture}
							\node (a) {};
							\node [below = 6cm of a] (b) {};
							\draw [-{Latex[length=5mm, width=5mm]}] (a) to node[right] {
									\begin{beamercolorbox}[wd=.8\textwidth,colsep=0.2cm,rounded=true,shadow=true]{hokudai}
										Search a rule with highest coverage under constraint of accuracy
										\small
										\begin{itemize}
											\item sample and update bounds $\Prec_u,\Prec_l$ unless
											      $\Prec(A)_u\le\tau$ or $\tau\le\Prec(A)_l$
										\end{itemize}
									\end{beamercolorbox}
								} (b);
						\end{tikzpicture}
						$A^*=\underset{
								A\in\mathcal{A}_t \ s.t. \ P(\Prec(A)\ge\tau)\ge1-\delta
							}{\arg\max}\operatorname{cov}(A)$
						\begin{tikzpicture}
							\node (a) {};
							\node [below = 6.0cm of a] (b) {};
							\node [below = 0.5cm of a] (c) {};
							\node [right = 13.0cm of c, align=left] (d) {$t\gets t+1$;\\continue;};
							\draw [-{Latex[length=5mm, width=5mm]}] (a) to node[below right] {
									if $A*$ is not null
								} (b);
							\draw [-{Latex[length=5mm, width=5mm]}] (c) to node[below] {
									if $A*$ is null
								} (d);
						\end{tikzpicture}

						\vspace{-0.8em}
						return $A^*$
					\end{column}
				\end{columns}
			\end{block}
		\end{column}
	\end{columns}
	\begin{columns}[t]
		\def\lcol{0.69}
		\def\rcol{0.30}
		\hspace{-1.0em}
		\begin{column}{\lcol\textwidth}
			\begin{block}{4. Experiments}
				\def\index{0012}
				\vspace{-0.4em}
				\begin{columns}[t]
					\begin{column}{.31\textwidth}
						\vspace{1.0em}
						\begin{figure}[tbp]
							\centering
							{
								\tiny
								\fontfamily{cmtt}\selectfont
								\begin{tabular}{p{10em}m{14em}}
									\toprule
									\csvreader[no head, late after line= \\]{%
										src/experiments/exp1/\index.csv
									}{}{%
									\ifnum\thecsvrow=16 \midrule\fi\csvcoli & \csvcolii % chktex 1
									}
									\bottomrule
								\end{tabular}
							}
							\vspace{1.0em}
							\caption{Focal point}
						\end{figure}
					\end{column}
					\begin{column}{.290\textwidth}
						\vspace{0.5em}
						\begin{figure}
							\includegraphics[width=\textwidth]{src/experiments/exp1/lime-\index}
							\caption{LIME's explanation}
						\end{figure}
						\vspace{0.2em}
						\begin{figure}
							\includegraphics[width=\textwidth]{src/experiments/exp1/anchor-\index-70}
							\vspace{-1.8em}
							\caption{Anchor's explanation ($\tau=0.70$)}
						\end{figure}
					\end{column}
					\begin{column}{.295\textwidth}
						\vspace{-0.4em}
						\begin{figure}
							\includegraphics[width=\textwidth]{src/experiments/exp1/rlime-\index-70}

							\vspace{-0.1em}
							\caption{R-LIME's explanation ($\tau=0.70$)}
						\end{figure}
						\vspace{0.8em}
						\begin{beamercolorbox}[colsep=0.1cm,rounded=true,shadow=true]{hokudai}
							\begin{itemize}
								\setlength{\itemsep}{0.3em}
								\item More interpretable and optimal than LIME
								\item More descriptive than Anchor
							\end{itemize}
							\vspace{0.5em}
						\end{beamercolorbox}
					\end{column}
				\end{columns}
				\begin{columns}
					\def\lcol{0.620}
					\def\rcol{0.295}
					\begin{column}{\lcol\textwidth}
						\vspace{-2.5em}
						\begin{columns}[t]
							\begin{column}{.509\textwidth}
								\begin{figure}
									\includegraphics[width=.88\textwidth]{src/experiments/exp2/box_plot}
									\vspace{-0.4em}
									\caption{LIME vs. R-LIME (in accuracy)}
								\end{figure}
							\end{column}
							\begin{column}{.491\textwidth}
								\begin{figure}
									\includegraphics[width=.88\textwidth]{src/experiments/exp2/comp_cov}
									\caption{LIME vs. R-LIME (in coverage)}
								\end{figure}
							\end{column}
						\end{columns}
					\end{column}
					\begin{column}{\rcol\textwidth}
						\vspace{0.5em}
						\begin{beamercolorbox}[colsep=0.1cm,rounded=true,shadow=true]{hokudai}
							\begin{itemize}
								\setlength{\itemsep}{0.3em}
								\setlength{\parskip}{0.2em}
								\item More accurate than LIME
								      \begin{itemize}
									      \item R-LIME adapts to optimized region flexibly
								      \end{itemize}
								\item More general than Anchor
								      \begin{itemize}
									      \item R-LIME captures decision boundary more plecisely
								      \end{itemize}
							\end{itemize}
						\end{beamercolorbox}
					\end{column}
				\end{columns}
			\end{block}
		\end{column}
		\begin{column}{\rcol\textwidth}
			\begin{block}{5. Conclusion}
				\renewcommand{\arraystretch}{1.4}
				\tabcolsep=0.6em
				\begin{center}
					\small
					\begin{tabular}{cccc}
						                    & LIME         & Anchor       & \textbf{R-LIME} \\
						\midrule
						Feature Importance  & \checkmark{} & $\times$     & \checkmark{}    \\
						Optimal Scope       & $\times$     & \checkmark{} & \checkmark{}    \\
						Interpretable Scope & $\times$     & \checkmark{} & \checkmark{}    \\
					\end{tabular}
				\end{center}

				% NOTE: center environment does not work here.
				%       we use columns environment instead.
				%       cf. https://tex.stackexchange.com/a/7441
				\vspace{0.5em}
				\begin{columns}
					\begin{column}{0.8\textwidth}
						\begin{beamercolorbox}[wd=\textwidth,colsep=0.3cm,rounded=true,shadow=true]{hokudai}
							\vspace{-0.3em}
							\begin{itemize}
								\item Achieved interpretability of both explanation and its scope! \\ [0.8em]
								      Also:
								      \begin{itemize}
									      \item More accurate than LIME
									      \item More general than Anchor
								      \end{itemize}
							\end{itemize}
						\end{beamercolorbox}~%
					\end{column}
				\end{columns}
			\end{block}
			\vspace{-0.6em}
			\begin{block}{References}
				\vspace{-0.6em}
				\begin{columns}[t]
					\begin{column}{0.99\textwidth}
						\printbibliography{}
					\end{column}
					\begin{column}{0.01\textwidth}
					\end{column}
				\end{columns}
				\vspace{-0.4em}
			\end{block}
		\end{column}
	\end{columns}
\end{frame}

\end{document}
